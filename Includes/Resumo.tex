% Resumo em l�ngua vern�cula
\begin{center}
	{\Large{\textbf{Roomie: uma aplica��o para pr�dios inteligentes baseada na infraestrutura de Internet das Coisas}}}
\end{center}

\vspace{1cm}

\begin{flushright}
	Autor: Viviane Costa Pinheiro\\
	Orientador(a): Prof. Me. Itamir de Barrocas Filho 
\end{flushright}

\vspace{1cm}

\begin{center}
	\Large{\textsc{\textbf{Resumo}}}
\end{center}

\noindent  
Com os avan�os tecnol�gicos dos �ltimos anos e a populariza��o da Internet das Coisas(IoT), hoje em dia � poss�vel desenvolver novas aplica��es que resolvem velhos problemas do nosso cotidiano que antigamente n�o possu�am uma solu��o vi�vel. Uma das �reas que est� sendo bastante explorada nesse novo contexto s�o as aplica��es espec�ficas para pr�dios inteligentes. Dentro dos problemas de gerenciamento de pr�dios, um dos problemas mais recorrentes � o mau uso dos espa�os, em especial das salas de reuni�o. No entanto os softwares tradicionais utilizados para reservas de salas n�o conseguem resolver todas as dificuldades relacionados a ger�ncia desses espa�os, principalmente por n�o se adaptarem ao ambiente no qual est�o inseridos e ao comportamento dos usu�rios em tempo real. Este trabalho tem como principal objetivo  descrever o desenvolvimento de uma aplica��o voltada para reserva de salas de reuni�o, a aplica��o utiliza infraestrutura de IoT para a incorpora��o de dados de sensores na aplica��o de reserva para permitir a adapta��o do sistema ao comportamento dos usu�rios e estados das salas. 
\noindent\textit{Palavras-chave}: Internet das Coisas,Pr�dios Inteligentes, Sistemas de Reserva de Sala.